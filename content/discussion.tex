%!TEX root = ../master_thesis.tex

\chapter{Discussion}
\label{chapter:discussion}

One of the starting points of this work was using fault trees as a way to model the dependability of microservice architectures. As a step towards constructing fault trees for a deployed microservice architecture, we investigated dependency graphs on the granularity level of applications and their service dependencies. During that process we learned that the dependency graphs themselves already are valuable tools in modeling the structure of a deployed architecture, since they allow engineers to answer the questions ``Which applications does my application depend on?'' and ``Which applications depend on my application?''. They even provide that visibility transitively over multiple degrees of service dependencies. In comparison, the qualitative fault trees we constructed were not better suited for visually assessing an architecture, due to their significantly larger size.

However, the structure of qualitative fault trees allowed us to construct quantitative fault trees for calculating TOP event probabilities. In this work we showed that it is possible to construct them, however the resulting TOP event probabilities were only a first step towards a wide range of opportunities for assessing dependability attributes. We especially see two fields of interest for future work: Quantitative fault trees may help to quantify the impact of changes to an architecture and therefore inform architectural and engineering decisions. They may also be valuable tools for aggregating realtime measurements when operating a microservice architecture.

One of the enabling factors in our investigations was the existence of infrastructure systems, which lead to a ``programmable infrastructure''. Due to their distributed nature, microservice architectures suffer from a distribution of meta information about the system. Gathering information about a deployed and running microservice architecture therefore requires dedicated infrastructure systems that collect, instrument, store and provide this meta information. These infrastructure systems are needed to allow automated investigations like the ones presented in this work.

This thesis is a first step towards researching the benefits of modeling dependability through dependencies in microservice architectures. To reduce the complexity of the work in this thesis, we excluded fault tolerance from investigations. We believe that including fault tolerance into the modeling approaches is a logical next step. This might aid representing reality better in the models and improve their expressiveness.