%!TEX root = ../master_thesis.tex

\newenvironment{tabstract}{
  \vspace*{\fill}
  \begin{center}%
    \bfseries\abstractname
  \end{center}
  }%
  {\vfill}

\pagebreak
\pagebreak

\selectlanguage{english}
\begin{tabstract}
The microservice architectural style is a common way of constructing software systems. Such a software system then consists of many applications that communicate with each other over a network. Each application has an individual software development lifecycle. To work correctly, applications depend on each other. In this thesis, we investigate how dependability of such a microservice architecture may be assessed. We specifically investigate how dependencies between applications may be modeled and how these models may be used to improve the dependability of a deployed microservice architecture. We evaluate dependency graphs as well as qualitative and quantitative fault trees as modeling approaches. For this work the author was embedded in the engineering team of ``SoundCloud'', a ``Software as a Service" company with 250 million monthly users. The modeling approaches were executed and qualitatively evaluated in the context of that real case study. As results we found that dependency graphs are a valuable tool for visualizing dependencies in microservice architectures. We also found quantitative fault trees to deliver promising results.
\end{tabstract}

\pagebreak

\selectlanguage{ngerman}
\begin{tabstract}
Der ``Microservice Architecture'' Stil ist eine verbreitete Art um Softwaresysteme zu erstellen. Solch ein Softwaresystem besteht aus vielen Applikationen, welche miteinander über ein Netzwerk kommunizieren. Jeder Applikation hat einen individuellen Softwareentwicklungszyklus. Um korrekt zu arbeiten verlassen sich die Applikationen aufeinander. In dieser Arbeit untersuchen wir, wie Verlässlichkeit von solch einer ``Microservice Architecture'' bewertet werden kann. Im Speziellen untersuchen wir, wie Abhängigkeiten zwischen Applikationen modelliert werden und diese Modelle dann benutzt werden können, um die Verlässlichkeit einer produktiven ``Microservice Architecture'' zu verbessern. Wir evaluieren die Modelierungsansätze Abhängigkeitsgraph sowie qualitativer und quantitativer Fehlerbaum. Für diese Arbeit war der Autor in die Ingenieurgruppe von ``SoundCloud'' eingebettet, einem ``Software as a Service''-Unternehmen mit 250 Millionen monatlichen Nutzern. Die Modelierungsansätze wurden im Kontext dieser echten Fallstudie ausgeführt sowie qualitativ ausgewertet. Als Resultate haben wir herausgefunden, dass Abhängigkeitsgraphen ein wertvolles Werkzeug zur Visualisierung von Abhängigkeiten in einer ``Microservice Architecture'' sind. Quantitative Fehlerbäume lieferten vielversprechende Resultate.
\end{tabstract}

\selectlanguage{english}